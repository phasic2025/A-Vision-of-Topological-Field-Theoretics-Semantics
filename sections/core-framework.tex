\section{Core Framework}

Having established the theoretical foundations, we now present the core mathematical framework for topological field-theoretic semantics. This section provides formal definitions, an axiomatic system, and fundamental principles that will guide the development in subsequent sections.

\subsection{Mathematical Foundations}

\subsubsection{Basic Definitions}

\begin{definition}[Semantic Manifold]
A \emph{semantic manifold} is a smooth manifold $M$ equipped with:
\begin{enumerate}
    \item A Riemannian metric $g$ encoding semantic distances.
    \item A connection $\nabla$ encoding semantic relationships and inference paths.
    \item A semantic field $\FieldOp{x}$ defined over $M$.
\end{enumerate}
We denote this structure as $(M, g, \nabla, \FieldOp{x})$.
\end{definition}

\begin{definition}[Semantic Field]
A \emph{semantic field} is a section $\FieldOp{x} \in \Gamma(E)$ of a vector bundle $E \to M$ over a semantic manifold $M$. The fiber $E_x$ at each point $x \in M$ represents the space of possible semantic values at that point.
\end{definition}

\begin{definition}[Topological Semantic Space]
A \emph{topological semantic space} is a pair $(\SemSpace{M}, \TopSpace{M})$ where:
\begin{enumerate}
    \item $\SemSpace{M}$ is a set of semantic objects (meanings, concepts, propositions, etc.).
    \item $\TopSpace{M}$ is a topology on $\SemSpace{M}$ such that semantic operations are continuous.
\end{enumerate}
\end{definition}

\subsubsection{Field Configurations and States}

\begin{definition}[Field Configuration]
A \emph{field configuration} is a complete specification of the semantic field $\FieldOp{x}$ at all points $x \in M$. The space of all field configurations is denoted $\mathcal{C}(M, E)$.
\end{definition}

\begin{definition}[Semantic State]
A \emph{semantic state} is a probability distribution (or quantum state) over field configurations, representing uncertainty or superposition of meanings.
\end{definition}

\subsubsection{Topological Invariants}

\begin{definition}[Semantic Homology]
The \emph{semantic homology groups} of a semantic space $\SemSpace{M}$ are the homology groups $H_k(\SemSpace{M})$ of the underlying topological space, computed with appropriate coefficients (e.g., $\mathbb{Z}$, $\mathbb{R}$, or field coefficients).
\end{definition}

\begin{definition}[Semantic Cohomology]
The \emph{semantic cohomology groups} $H^k(\SemSpace{M})$ provide dual information, often encoding semantic operations, transformations, and obstructions.
\end{definition}

\subsection{Axiomatic System}

We propose the following axioms as fundamental principles of topological field-theoretic semantics:

\begin{axiom}[Continuity of Meaning]
Semantic operations (composition, inference, abstraction) are continuous maps with respect to the semantic topology.
\end{axiom}

\begin{axiom}[Locality]
The semantic field at a point $x$ depends only on the field values in a neighborhood of $x$ (with appropriate decay conditions).
\end{axiom}

\begin{axiom}[Topological Invariance]
Topological invariants of semantic spaces are preserved under continuous semantic transformations.
\end{axiom}

\begin{axiom}[Field Dynamics]
Semantic fields evolve according to field equations derived from a semantic action principle.
\end{axiom}

\begin{axiom}[Superposition]
In contexts allowing ambiguity or multiple interpretations, semantic states can superpose, with probabilities (or amplitudes) given by field-theoretic path integrals.
\end{axiom}

\begin{axiom}[Conservation]
Certain semantic properties are conserved under semantic transformations, corresponding to topological or symmetry invariants.
\end{axiom}

\subsection{Fundamental Principles}

\subsubsection{Principle of Topological Equivalence}

Semantic spaces with the same topological invariants (homotopy type, homology groups, etc.) should be considered equivalent for the purposes of semantic analysis, even if they differ in their specific geometric realization.

\subsubsection{Principle of Field-Theoretic Dynamics}

Semantic evolution is governed by extremizing a semantic action $\Action[\FieldOp{x}]$, leading to field equations of the form:
\begin{equation}
    \frac{\delta \Action}{\delta \FieldOp{x}} = 0
\end{equation}
where $\delta/\delta\FieldOp{x}$ denotes the functional derivative.

\subsubsection{Principle of Topological Constraints}

The topology of the base space imposes constraints on possible field configurations. These constraints manifest as:
\begin{itemize}
    \item Boundary conditions at topological boundaries.
    \item Quantization conditions for fields on compact spaces.
    \item Obstructions preventing certain field configurations.
\end{itemize}

\subsubsection{Principle of Semantic Gauge Invariance}

Semantic representations may exhibit gauge freedom---multiple mathematically distinct representations that are semantically equivalent. The physical (semantic) content should be gauge-invariant.

\subsubsection{Principle of Scale Separation}

Semantic structure exists at multiple scales:
\begin{itemize}
    \item \textbf{Microscopic}: Individual words, concepts, atomic meanings.
    \item \textbf{Macroscopic}: Documents, discourses, large-scale semantic structures.
    \item \textbf{Effective theories}: Coarse-grained descriptions valid at each scale.
\end{itemize}

\subsubsection{Principle of Semantic Renormalization}

As we change the scale of analysis, semantic theories must be renormalized---systematic procedures for removing scale-dependent artifacts while preserving universal, scale-invariant properties.

\subsection{Mathematical Structure}

The framework exhibits rich mathematical structure:

\begin{itemize}
    \item \textbf{Category theory}: Semantic spaces and field configurations form categories, with functors encoding semantic operations.
    \item \textbf{Differential geometry}: Semantic manifolds support differential geometric structures (metrics, connections, curvature).
    \item \textbf{Algebraic topology}: Homology, cohomology, and homotopy theory provide invariants and classification.
    \item \textbf{Functional analysis}: Field configurations live in function spaces with appropriate topologies.
    \item \textbf{Quantum field theory}: Path integrals, Feynman diagrams, and renormalization techniques apply.
\end{itemize}

This mathematical richness enables both rigorous theoretical development and practical computational implementation, as we will see in subsequent sections.

