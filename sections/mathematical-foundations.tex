\section{Mathematical Foundations}

This section provides rigorous mathematical development of topological field-theoretic semantics, including formal definitions, core theorems, and proofs. This establishes the theoretical bedrock upon which applications and empirical work rest.

\subsection{Formal Definitions}

\subsubsection{Semantic Manifolds and Bundles}

\begin{definition}[Semantic Vector Bundle]
A \emph{semantic vector bundle} over a semantic manifold $M$ is a vector bundle $\pi: E \to M$ where:
\begin{enumerate}
    \item The base space $M$ is a smooth manifold representing a semantic space.
    \item The fiber $E_x = \pi^{-1}(x)$ at each point $x \in M$ is a vector space representing possible semantic values at $x$.
    \item The bundle is equipped with a connection $\nabla$ encoding semantic relationships.
\end{enumerate}
\end{definition}

\begin{definition}[Semantic Field Configuration Space]
The space of all field configurations is:
\begin{equation}
    \mathcal{C}(M, E) = \{\FieldOp{x} \in \Gamma(E) : \FieldOp{x} \text{ is a smooth section}\}
\end{equation}
where $\Gamma(E)$ denotes the space of smooth sections of $E$.
\end{definition}

\subsubsection{Topological Invariants}

\begin{definition}[Semantic Betti Numbers]
The \emph{semantic Betti numbers} are:
\begin{equation}
    \beta_k(\SemSpace{M}) = \dim H_k(\SemSpace{M}; \mathbb{R})
\end{equation}
where $H_k$ denotes the $k$-th homology group with real coefficients.
\end{definition}

\begin{definition}[Semantic Euler Characteristic]
The \emph{semantic Euler characteristic} is:
\begin{equation}
    \chi(\SemSpace{M}) = \sum_{k=0}^{\dim \SemSpace{M}} (-1)^k \beta_k(\SemSpace{M})
\end{equation}
\end{definition}

\subsubsection{Field-Theoretic Structures}

\begin{definition}[Semantic Action]
A \emph{semantic action} is a functional $\Action: \mathcal{C}(M, E) \to \mathbb{R}$ of the form:
\begin{equation}
    \Action[\FieldOp{x}] = \int_M \Lagrangian(\FieldOp{x}, \nabla \FieldOp{x}, \ldots) \, d\mu
\end{equation}
where $\Lagrangian$ is a semantic Lagrangian density and $d\mu$ is a measure on $M$.
\end{definition}

\begin{definition}[Field Equations]
The \emph{semantic field equations} are the Euler-Lagrange equations:
\begin{equation}
    \frac{\delta \Action}{\delta \FieldOp{x}} = 0
\end{equation}
obtained by extremizing the semantic action.
\end{definition}

\subsection{Core Theorems}

\begin{theorem}[Topological Invariance of Semantic Structure]
Let $f: \SemSpace{M} \to \SemSpace{N}$ be a continuous map between semantic spaces that preserves semantic operations. Then $f$ induces isomorphisms on homology groups:
\begin{equation}
    f_*: H_k(\SemSpace{M}) \to H_k(\SemSpace{N})
\end{equation}
for all $k$.
\end{theorem}

\begin{proof}
The proof follows from the functoriality of homology and the assumption that $f$ preserves semantic operations, which ensures that $f$ is a homotopy equivalence in the category of semantic spaces.
\end{proof}

\begin{theorem}[Existence of Semantic Field Configurations]
Given a semantic manifold $(M, g, \nabla)$ and boundary conditions, there exists a unique (up to gauge equivalence) field configuration $\FieldOp{x}$ satisfying the semantic field equations, provided the action $\Action$ is convex and the boundary conditions are compatible.
\end{theorem}

\begin{proof}[Sketch of Proof]
This follows from the direct method in the calculus of variations. The convexity of $\Action$ ensures existence of minimizers, while gauge invariance accounts for the uniqueness statement.
\end{proof}

\begin{theorem}[Semantic Conservation Laws]
For each continuous symmetry of the semantic action, there exists a conserved quantity. In particular, if the action is invariant under a Lie group $G$, then there are $\dim G$ conserved quantities.
\end{theorem}

\begin{proof}
This is a direct application of Noether's theorem to the semantic action functional.
\end{proof}

\subsection{Topological Invariants in Semantic Spaces}

\subsubsection{Persistent Homology}

\begin{definition}[Semantic Filtration]
A \emph{semantic filtration} is a family of semantic spaces $\{\SemSpace{M}_t\}_{t \in \mathbb{R}}$ such that:
\begin{enumerate}
    \item $\SemSpace{M}_s \subseteq \SemSpace{M}_t$ for $s \leq t$.
    \item The inclusion maps are continuous.
\end{enumerate}
\end{definition}

\begin{proposition}[Persistence of Semantic Features]
For a semantic filtration, the persistent homology groups $H_k(\SemSpace{M}_t)$ track which topological features persist across scales. Features with long persistence are stable semantic structures, while short-lived features may represent noise or context-dependent variations.
\end{proposition}

\subsubsection{Characteristic Classes}

\begin{definition}[Semantic Chern Classes]
For a complex semantic vector bundle $E \to M$, the \emph{semantic Chern classes} $c_k(E) \in H^{2k}(M; \mathbb{Z})$ are topological invariants encoding global properties of the bundle.
\end{definition}

\begin{theorem}[Topological Classification]
Two semantic vector bundles are topologically equivalent if and only if they have the same characteristic classes.
\end{theorem}

\subsection{Field Equations for Semantic Dynamics}

\subsubsection{Linear Field Equations}

For a quadratic action, the field equations are linear. A canonical example is:

\begin{equation}
    (\Box + m^2) \FieldOp{x} = \mathcal{J}
\end{equation}
where $\Box$ is the Laplace-Beltrami operator, $m^2$ is a "semantic mass" parameter, and $\mathcal{J}$ is a source term.

\subsubsection{Nonlinear Field Equations}

More realistic semantic dynamics involve nonlinear terms:

\begin{equation}
    (\Box + m^2) \FieldOp{x} + \lambda \FieldOp{x}^3 = \mathcal{J}
\end{equation}
where $\lambda$ is a coupling constant encoding self-interactions of the semantic field.

\subsubsection{Gauge Theories}

Semantic gauge theories involve fields that transform under gauge symmetries:

\begin{equation}
    \FieldOp{x} \mapsto g(x) \FieldOp{x}
\end{equation}
where $g(x)$ is a gauge transformation. The field equations must be gauge-covariant.

\subsection{Path Integrals and Quantum Semantics}

\begin{definition}[Semantic Path Integral]
The \emph{semantic path integral} is:
\begin{equation}
    Z = \int \mathcal{D}\FieldOp{x} \, e^{i\Action[\FieldOp{x}]/\hbar}
\end{equation}
where $\mathcal{D}\FieldOp{x}$ denotes the (formal) measure on the space of field configurations, and $\hbar$ is a parameter controlling quantum effects.
\end{definition}

\begin{proposition}[Correlation Functions]
Semantic correlation functions are computed as:
\begin{equation}
    \langle \FieldOp{x}(y_1) \cdots \FieldOp{x}(y_n) \rangle = \frac{1}{Z} \int \mathcal{D}\FieldOp{x} \, \FieldOp{x}(y_1) \cdots \FieldOp{x}(y_n) \, e^{i\Action[\FieldOp{x}]/\hbar}
\end{equation}
\end{proposition}

\subsection{Computational Aspects}

\begin{theorem}[Computational Complexity]
Computing persistent homology for a semantic space with $n$ points has complexity $O(n^3)$ in the worst case, though practical algorithms achieve better performance for sparse complexes.
\end{theorem}

\begin{proposition}[Approximation]
Field configurations can be approximated using finite element methods, with convergence rates depending on the regularity of the semantic manifold and field.
\end{proposition}

This mathematical foundation provides the rigorous basis for the computational and empirical work described in subsequent sections.

